% arara: xelatex
\documentclass[12pt]{article}

% \usepackage{physics}

\usepackage{hyperref}
\hypersetup{
    colorlinks=true,
    linkcolor=blue,
    filecolor=magenta,      
    urlcolor=cyan,
    pdftitle={Overleaf Example},
    pdfpagemode=FullScreen,
    }

\usepackage{tikzducks}

\usepackage{tikz} % картинки в tikz
\usetikzlibrary{shapes, arrows, positioning}
\usepackage{microtype} % свешивание пунктуации

\usepackage{array} % для столбцов фиксированной ширины

\usepackage{indentfirst} % отступ в первом параграфе

\usepackage{sectsty} % для центрирования названий частей
\allsectionsfont{\centering}

\usepackage{amsmath, amsfonts, amssymb} % куча стандартных математических плюшек

\usepackage{comment}

\usepackage[top=2cm, left=1.2cm, right=1.2cm, bottom=2cm]{geometry} % размер текста на странице

\usepackage{lastpage} % чтобы узнать номер последней страницы

\usepackage{enumitem} % дополнительные плюшки для списков
%  например \begin{enumerate}[resume] позволяет продолжить нумерацию в новом списке
\usepackage{caption}

\usepackage{url} % to use \url{link to web}


\newcommand{\smallduck}{\begin{tikzpicture}[scale=0.3]
    \duck[
        cape=black,
        hat=black,
        mask=black
    ]
    \end{tikzpicture}}

\usepackage{fancyhdr} % весёлые колонтитулы
\pagestyle{fancy}
\lhead{ICEF, Stochastic calculus}
\chead{}
\rhead{Retake exam, 2025-01-25}
\lfoot{}
\cfoot{}
\rfoot{}

\renewcommand{\headrulewidth}{0.4pt}
\renewcommand{\footrulewidth}{0.4pt}

\usepackage{tcolorbox} % рамочки!

\usepackage{todonotes} % для вставки в документ заметок о том, что осталось сделать
% \todo{Здесь надо коэффициенты исправить}
% \missingfigure{Здесь будет Последний день Помпеи}
% \listoftodos - печатает все поставленные \todo'шки


% более красивые таблицы
\usepackage{booktabs}
% заповеди из докупентации:
% 1. Не используйте вертикальные линни
% 2. Не используйте двойные линии
% 3. Единицы измерения - в шапку таблицы
% 4. Не сокращайте .1 вместо 0.1
% 5. Повторяющееся значение повторяйте, а не говорите "то же"


\setcounter{MaxMatrixCols}{20}
% by crazy default pmatrix supports only 10 cols :)


\usepackage{fontspec}
\usepackage{libertine}
\usepackage{polyglossia}

\setmainlanguage{russian}
\setotherlanguages{english}

% download "Linux Libertine" fonts:
% http://www.linuxlibertine.org/index.php?id=91&L=1
% \setmainfont{Linux Libertine O} % or Helvetica, Arial, Cambria
% why do we need \newfontfamily:
% http://tex.stackexchange.com/questions/91507/
% \newfontfamily{\cyrillicfonttt}{Linux Libertine O}

\AddEnumerateCounter{\asbuk}{\russian@alph}{щ} % для списков с русскими буквами
% \setlist[enumerate, 2]{label=\asbuk*),ref=\asbuk*}

%% эконометрические сокращения
\DeclareMathOperator{\Cov}{\mathbb{C}ov}
\DeclareMathOperator{\Corr}{\mathbb{C}orr}
\DeclareMathOperator{\Var}{\mathbb{V}ar}
\DeclareMathOperator{\col}{col}
\DeclareMathOperator{\row}{row}

\let\P\relax
\DeclareMathOperator{\P}{\mathbb{P}}

\DeclareMathOperator{\E}{\mathbb{E}}
% \DeclareMathOperator{\tr}{trace}
\DeclareMathOperator{\card}{card}

\DeclareMathOperator{\Convex}{Convex}
\DeclareMathOperator{\plim}{plim}

\newcommand{\cN}{\mathcal{N}}
\newcommand{\cF}{\mathcal{F}}

\newcommand{\RR}{\mathbb{R}}
\newcommand{\NN}{\mathbb{N}}
\newcommand{\hb}{\hat{\beta}}
\newcommand{\dPois}{\mathrm{Pois}}


\newcommand{\dExpo}{\mathrm{Expo}}

\renewcommand*{\thefootnote}{\fnsymbol{footnote}}

\usepackage{mathtools}
\DeclarePairedDelimiter{\norm}{\lVert}{\rVert}
\DeclarePairedDelimiter{\abs}{\lvert}{\rvert}
\DeclarePairedDelimiter{\scalp}{\langle}{\rangle}
\DeclarePairedDelimiter{\ceil}{\lceil}{\rceil}

\begin{document}

Rules: 120 minutes, one A4 cheat sheet and calculator is ok, $(W_t)$ denotes a Wiener process.
You may use the standard normal cumulative distribution function $F()$ in your answers.

\begin{enumerate}
    \item {[10]} Let $(W_t)$ be a standard Wiener process. 
    \begin{enumerate}
        \item {[3]} What is the distribution of $3W_7 + W_8$? 
        % \item {[2]} Find the probability $\P(W_7 + 3W_8 > 1)$ in terms of a standard normal cdf $F()$.
        \item {[5]} What is the conditional distribution of $(3W_7 + W_8 \mid W_1 = 2)$?
        \item {[2]} Find the probability $\P(3W_7 + W_8 > 1 \mid W_1 = 2)$.
    \end{enumerate}


    %\item {[10]} Consider the processes $X_t = t + \int_0^t (W_u^3 + W_u) \, dW_u$.
    %\begin{enumerate}
        %\item {[2]} Find $dS_t$.
        %\item {[2]} Is $(S_t)$ a martingale? Why?
%        \item {[2 + 3]} Find $\E(X_t)$ and $\Var(X_t)$.
%        \item {[5]} Find $\Cov(X_t, W_t)$.
%    \end{enumerate}


    \item {[10]} Consider the process $Y_t = t + \int_0^t u W_u  du$.
     \begin{enumerate}
    %     %\item {[2]} Find $dS_t$.
    %     %\item {[2]} Is $(S_t)$ a martingale? Why?
         \item {[2 + 3]} Find $\E(Y_t)$ and $\Var(Y_t)$.
         \item {[5]} Find $\Cov(Y_t, W_t)$.
    \end{enumerate}

    Hint: do not forget about Ito's lemma :)


    \item {[10]} Let $M_t = h(t) \cdot \cos W_t$.
    \begin{enumerate}
        \item {[6]} Find a non-zero function $h(t)$ such that $M_t$ is a martingale.
        \item {[4]} Find $\E(\cos W_t)$.
    \end{enumerate}
    

    \item {[10]} Consider the Black and Scholes model with riskless rate $r$, volatility $\sigma$ and initial share price $S_0$. 

Measure $\P^*$ denotes the risk-neutral probability that makes discounted share price process a martingale and 
measure $\P$ denotes the probability under which $(W_t)$ is a Wiener process and $dS_t = \mu S_t dt + \sigma S_t dW_t$.

\begin{enumerate}
         \item {[2 + 2]} Find $\P^*(S_T > \exp(1))$ and $\P(S_T > \exp(1))$.
         \item {[6]} Find the current price $X_0$ of an option that pays you $X_T = \max\{1, - \ln S_T\}$ at fixed time $T$.
     \end{enumerate}
        

    \item {[10]} Let $dX_t = X_t dt + X_t dW_t$ with $X_0 = 0$.
    
    Find at least one solution of this stochastic differential equation. 

    \item {[10]} The random variables $X_1$, $X_2$, \dots{ } are independent and exponentially distributed with rate $\lambda  = 1$.
    Let $Y_n = \max\{X_1, X_2, \dots, X_n\}$.

    \begin{enumerate}
        \item {[2]} Provide an example of event $A$ such that $A \in \sigma(Y_5)$ but $A \notin \sigma(X_5)$. 
        \item {[4 + 4]} Find $\E(Y_{n+1} \mid Y_n)$ and $\Var(Y_{n+1} \mid Y_n)$.
    \end{enumerate}

Note: here $\sigma(.)$ is a minimal sigma-algebra generated by random variable.

    % \item {[10]}
% \item мартингал в непрерывном времени? дуб?


\end{enumerate}


\end{document}

